% arara: lualatex: {options: []}
% arara: biber: { options: [ '--wraplines' ] }
% arara: lualatex: {options: []}
\documentclass[10pt]{article}
\usepackage{notestex,csquotes,tabularx,lipsum,calc}
\SetColorProfile{243EB3}{515E98}{B14523}{68B324}{A4C388}
\usepackage[ngerman]{babel}
\usepackage[
backend=biber,
style=alphabetic,
sorting=ynt
]{biblatex}
\allowdisplaybreaks

\usepgfplotslibrary{fillbetween}
\usetikzlibrary{patterns}

\begin{document}
\title{{Mathematik}\\{\normalsize{\itshape Leistungskurs}}}
\author{Never Gude}
\affiliation{
Carl-Friedrich-Gauß Gymnasium Hockenheim\\
\href{https://github.com/minced1}{GitHub}\\
}
\emailAdd{j.gude@posteo.de}
\maketitle
\newpage
\pagestyle{fancynotes}

\part{Analytische Geometrie}
\section{Modellieren von geradlinigen Bewegungen}
\begin{definition}
	Eine geradlinige Bewegung kann in Form einer Geraden modelliert werden.\\
	Hierbei ist $\vec{x}$ die Position zum Zeitpunkt $t$. Der Körper startet bei $t=0$ in Punkt $\vec{p}$ und bewegt sich in Richtung $\vec{v}$ mit Geschwindigkeit $|\vec{v}|$.
	\[
	\vec{x} = \vec{p} + t \cdot \vec{v}
	\]
\end{definition}

\begin{example}
\[
\]
\end{example}

\begin{lstlisting}[
	language=Python,
	style=standalone
	]
import numpy as np

def incmatrix(genl1,genl2):
string = "Hello World"
m = len(genl1)
n = len(genl2)
M = None #to become the incidence matrix dsjklasd dsa dsadasda da dsadadadas
VT = np.zeros((n*m,1), int)  #dummy variable

#compute the bitwise xor matrix
M1 = bitxormatrix(genl1)
M2 = np.triu(bitxormatrix(genl2),1)

for i in range(m-1):
	for j in range(i+1, m):
		[r,c] = np.where(M2 == M1[i,j])
		for k in range(len(r)):
			VT[(i)*n + r[k]] = 1;
			VT[(i)*n + c[k]] = 1;
			VT[(j)*n + r[k]] = 1;
			VT[(j)*n + c[k]] = 1;

			if M is None:
				M = np.copy(VT)
			else:
				M = np.concatenate((M, VT), 1)

			VT = np.zeros((n*m,1), int)

return M

\end{lstlisting}

\newpage
\part{Stochastik}
\section{Elementare Kombinatorik}
\begin{definition}
Aus $n$ Elementen werden $k$ ausgewählt. Die \textbf{Anzahl der Möglichkeiten}
\begin{center}
\begin{tabular}[c]{ccc}
mit Reihenfolge & mit Reihenfolge & ohne Reihenfolge\\
mit Wiederholung & ohne Wiederholung & ohne Wiederholung\\
$n^k$ & $\frac{n!}{(n-k)!}$ & $\binom{n}{k} = \frac{n!}{(n-k)! \cdot k!}$
\end{tabular}
\end{center}
\end{definition}

\section{Pfadregeln}
Beim Wiederholen eines Spiels, z. B. beim mehrfachen Ziehen einer Kugel aus einer Urne oder mehrfachen Würfeln, werden die Wahrscheinlichkeiten jedes Durchgangs \enquote{aneinander gereiht}.\\
Beim traversieren des Baumes in die Tiefe wird multipliziert, beim traversieren in die Breite wird addiert


\begin{marginfigure}
	\centering
	\usetikzlibrary {graphs,graphdrawing} \usegdlibrary {trees}
	\tikz [binary tree layout, level distance=5mm]
	\node[circle, fill=black] {}
	child { node {$\frac{1}{4}$}
		child { node {$\frac{1}{4}$}}
		child { node {$\frac{3}{4}$}}
	}
	child { node {$\frac{3}{4}$}
		child { node {$\frac{1}{4}$}}
		child { node {$\frac{3}{4}$}}
	};
	\caption{Binärbaum}
\end{marginfigure}

\begin{definition}
Beim Berechnen der Wahrscheinlichkeit eines Ereignisses $E$ (anhand eines Baums), das aus mehreren Ergebnissen besteht, gelten folgende Regeln\\
\begin{itemize}[leftmargin=6pt+\widthof{\bfseries Summenregel}]
	\item[\bfseries Produktregel] Für Wahrscheinlichkeit eines Erbegnisses, alle Wahrscheinlichkeiten des Pfades multiplizieren
	\item[\bfseries Summenregel] Für Wahrscheinlichkeit eines Ereignisses $E$, alle Wahrscheinlichkeiten der zu $E$ gehörenden Ergebnisse addieren
\end{itemize}
\end{definition}
\newpage
\section{Erwartungswert, Varianz\\ und Standardabweichung}
\begin{definition}
Die Zufallsgröße $X$ kann die Werte $x_1, x_2, \dots, x_n$ annehmen.\\
Der \textbf{Erwartungswert} $E(X)$ ist definiert als Wert, der für $X$ auf lange Sicht zu erwarten ist.
\[
E(X)=\sum_{i=1}^{n} x_i \cdot P(X = x_i)
\]
\begin{center}
	Ein Spiel ist fair, wenn $E(X)=0$ ist
\end{center}
Die \textbf{Varianz} $Var(X)$ ist definiert als Maß für die Streuung der Werte um den Erwartungswert $E(X)$. Dabei wird die mittlere quadratische Abweichung:
\[
Var(X)=\sum_{i=1}^n(x_i-E(X))^2\cdot P(X=x_i)
\]
Um Einheiten wie $€^2$ zu vermeiden wird für die \textbf{Standardabweichung} $\sigma$ die Wurzel aus $Var(X)$ gezogen:
\[
\sigma=\sqrt{Var(X)}=\sqrt{\sum_{i=1}^n(x_i-E(X))^2\cdot P(X=x_i)}
\]
\end{definition}

\section{Binomialverteilte Zufallsvariablen}
Ein Bernoulli-Experiment ist ein Zufallsexperiment, dass als Ergebnisse nur \textit{Treffer} und \textit{Nicht-Treffer} kennt. Die Trefferwahrscheinlichkeit wird mit $p$ angegeben.\\
Bei $n$ unabhängigen Wiederholung spricht man von einer Bernoulli-Kette der Länge $n$.
\begin{definition}
Die Wahrscheinlichkeit einer \textbf{Bernoulli-Kette} der Länge $n$ mit Trefferwahrscheinlichkeit $p$ für genau $k$ Treffer wird wie folgt berechnet:
\[
P_{p}^{n}(X=k) = \binom{n}{k} \cdot p^k \cdot (1-p)^{n-k}
\]

Für höchstens $k$ Treffer gilt folgendes:
\[
P_{p}^{n}(X\leq k) = \sum_{i=0}^k P_{p}^{n}(X = i)
\]
\end{definition}

\begin{remark}
Berechnung mit WTR:
\begin{itemize}[leftmargin=6em]
	\item[$P_{p}^{n}(X=k)$:] \texttt{2nd\rightarrow stat-reg/distr\rightarrow DISTR\rightarrow Binomialpdf}
	\item[$P_{p}^{n}(X\leq k)$:] \texttt{2nd\rightarrow stat-reg/distr\rightarrow DISTR\rightarrow Binomialcdf}
\end{itemize}
\end{remark}


\section{Bedingte Wahrscheinlichkeit}
\begin{margintable}
\centering
\resizebox{\linewidth}{!}{
\begin{tabular}{|c|c|c|c|}
	\hline
	& $B$ & $\overline{B}$ & \\\hline
	$A$ & $P(A\cap B)$ & $P(A\cap \overline{B})$ & $P(A)$\\\hline
	$\overline{A}$ & $P(\overline{A}\cap B)$ & $P(\overline{A}\cap \overline{B})$ & $P(\overline{A})$\\\hline
	& $P(B)$ & $P(\overline{B})$ & $1$\\\hline
	\end{tabular}
}
\caption{Vierfeldertafel}
\end{margintable}


\begin{definition}
	Für zwei Ergebnisse $A$ und $B$ mit $P(A)\neq 0$ ist $P_A(B)$ die \textbf{bedingte Wahrscheinlichkeit} für das Eintreten von $B$ unter der Bedingung, dass $A$ eingetreten ist.
	\[
	P_A(B)=\frac{P(A\cap B)}{P(A)}
	\]
\end{definition}
\begin{definition}
	Zwei Ergebnisse $A$ und $B$ mit $P(A)\neq 0$ sind \textbf{stochastisch unabhängig}, wenn gilt:
	\[
	P(A)\cdot P(B)=P(A\cap B)
	\]
	\centering{also}
	\[
	P_A(B)=P(B)
	\]
\end{definition}

\section{Erwartungswert, Varianz\\ und Standardabweichung bei Binomialverteilung}
\begin{definition}
Ist eine Zufallsgröße binomialverteilt, also $B_{n,p}$-verteilt, so ist
	\begin{itemize}[leftmargin=6pt+\widthof{\bfseries Standardabweichung}]
	\item[\bfseries Erwartungswert] $\mu = n\cdot p$
	\item[\bfseries Varianz] $Var(X)=n\cdot p\cdot (1-p)$
	\item[\bfseries Standardabweichung] $\sigma=\sqrt{n\cdot p\cdot (1-p)}$
	\end{itemize}
\end{definition}

\newpage
\section{Testen von Hypothesen}

\subsection{Einseitiger Hypothesentest}
\begin{example}
Ein Betrieb bezieht Bauteile von einer Zulieferfirma. Der Betriebsleiter vermutet,
dass mehr als 5\% der Bauteile defekt sind. Dazu führt er einen Hypothesentest
mit einem Stichprobenumfang von 200 Bauteilen auf dem Signifikanzniveau 5\% durch. Als Nullhypothese wählt er:

\enquote{Höchstens 5\% der Bauteile sind defekt}.
\paragraph{a)} Bestimmen Sie den Ablehnungsbereich und formulieren Sie die Entscheidungsregel.
\addtolength{\jot}{1em}
\begin{align*}
&H_0 \leq 0,05 &\parbox[t]{0.5\linewidth}{Höchstens 5\% der Bauteile sind defekt; D. h. Rechtsseitiger Test} \\
&H_1 > 0,05 &\parbox[t]{0.5\linewidth}{Mehr als 5\% der Bauteile sind defekt} \\
&n = 200; \hat\alpha = 0,05 &\parbox[t]{0.5\linewidth}{Stichprobenumfang; Signifikanzniveau} \\
&\overline A = [g;\dots;n] &\parbox[t]{0.5\linewidth}{Ablehnungsbereich $\overline A$} \\
&P^n_p(X \geq g) \leq \hat\alpha &\parbox[t]{0.5\linewidth}{Rechtsseitigen Test durchführen} \\
&1 - P^{200}_{0,05}(X \leq g - 1) \leq 0,05 &\parbox[t]{0.5\linewidth}{} \\
&P^{200}_{0,05}(X \leq g - 1) \geq 0,95 &\parbox[t]{0.5\linewidth}{Gesucht ist kleinste natürliche Zahl $g$, die Kriterium erfüllt} \\
& \begin{tabular}{|c|c|}
	\hline
	$P^{200}_{0,05}(X \leq g - 1)$ & $g$ \\ \hline
	$0,976$ & $17$ \\
	$0,956$ & $16$ \\
	$0,922$ & $15$ \\
	\hline
	\end{tabular} &\parbox{0.5\linewidth}{Per \texttt{Binomialcdf} ausprobieren; $g = 16$ ist kleinste natürliche Zahl, die Kriterium erfüllt}\\
&\overline A = [16;\dots;200] &\parbox[t]{0.5\linewidth}{Wenn das Stichprobenergebnis mehr als $15$ defekte Bauteile liefert, wird $H_0$ verworfen, sonst nicht.}
\end{align*}
\paragraph{b)} Geben Sie die Irrtumswahrscheinlichkeit an.
\begin{align*}
&1 - P^{200}_{0,05}(X \leq g - 1) \leq 0,05 &\parbox[t]{0.5\linewidth}{Fehler 1. Art berechnen} \\
&1 - P^{200}_{0,05}(X \leq 16 - 1) \leq 0,05 &\parbox[t]{0.5\linewidth}{Gesucht ist kleinste natürliche Zahl $g$, die Kriterium erfüllt} \\
&1 - 0,956 \leq 0,05 &\parbox[t]{0.5\linewidth}{} \\
&0,044 \leq 0,05 &\parbox[t]{0.5\linewidth}{Irrtumswahrscheinlichkeit ist $0,044$}
\end{align*}
\end{example}

\subsection{Zweiseitiger Hypothesentest}
\begin{example}
Bei einer leicht verbeulten Münze wird vermutet, dass \enquote{Wappen} nicht mit einer
Wahrscheinlichkeit von 50\% fällt. Dazu soll ein Hypothesentest durchgeführt werden, bei dem
die Münze 100-mal geworfen wird. Als Signifikanzniveau wird 5\% gewählt.

Bestimmen Sie den Ablehnungsbereich und formulieren Sie die Entscheidungsregel

\addtolength{\jot}{1em}
\begin{align*}
&H_0 = 0,5 &\parbox[t]{0.5\linewidth}{\enquote{Wappen} fällt in 50\% der Fälle} \\
&H_1 \neq 0,5 &\parbox[t]{0.5\linewidth}{\enquote{Wappen} fällt nicht in 50\% der Fälle} \\
&n = 100; \hat\alpha = 0,05 &\parbox[t]{0.5\linewidth}{Stichprobenumfang; Signifikanzniveau} \\
&\overline A = [0;\dots;g_1] \cup [g_2;\dots;n] &\parbox[t]{0.5\linewidth}{Ablehnungsbereich $\overline A$} \\
&P^n_p(X \leq g_1) \leq \frac{\hat\alpha}{2} &\parbox[t]{0.5\linewidth}{Linksseitigen Test durchführen} \\
&P^{100}_{0,5}(X \leq g_1) \leq 0,025 &\parbox[t]{0.5\linewidth}{Gesucht ist größte natürliche Zahl $g_1$, die Kriterium erfüllt} \\
& \begin{tabular}{|c|c|}
	\hline
	$P^{100}_{0,5}(X \leq g_1)$ & $g_1$ \\ \hline
	$0,01$ & $38$ \\
	$0,018$ & $39$ \\
	$0,028$ & $40$ \\
	\hline
	\end{tabular} &\parbox{0.5\linewidth}{Per \texttt{Binomialcdf} ausprobieren; $g_1 = 39$ ist größte natürliche Zahl, die Kriterium erfüllt}\\
&P^n_p(X \geq g_2) \leq \frac{\hat\alpha}{2} &\parbox[t]{0.5\linewidth}{Rechtsseitigen Test durchführen} \\
&1 - P^{100}_{0,5}(X \leq g_2 - 1) \leq 0,025 &\parbox[t]{0.5\linewidth}{} \\
&P^{100}_{0,5}(X \leq g_2 - 1) \geq 0,975 &\parbox[t]{0.5\linewidth}{Gesucht ist kleinste natürliche Zahl $g_2$, die Kriterium erfüllt} \\
& \begin{tabular}{|c|c|}
	\hline
	$P^{100}_{0,5}(X \leq g_2 - 1)$ & $g_2$ \\ \hline
	$0,99$ & $62$ \\
	$0,982$ & $61$ \\
	$0,972$ & $60$ \\
	\hline
	\end{tabular} &\parbox{0.5\linewidth}{Per \texttt{Binomialcdf} ausprobieren; $g_2 = 61$ ist kleinste natürliche Zahl, die Kriterium erfüllt}\\
&\overline A = [0;\dots;39] \cup [61;\dots;100] &\parbox[t]{0.5\linewidth}{Wenn das Stichprobenergebnis weniger als $40$ oder mehr als $60$ mal \enquote{Wappen} liefert, wird $H_0$ verworfen, sonst nicht.}
\end{align*}
\end{example}

\newpage
\section{Stetige Zufallsgrößen}
\subsection{Normalverteilung}
\pgfmathdeclarefunction{gauss}{2}{%
  \pgfmathparse{1/(#2*sqrt(2*pi))*exp(-((x-#1)^2)/(2*#2^2))}%
}
\begin{example}
Kartoffelchips werden in Tüten gefüllt. Die Masse (in Gramm) des Tüteninhalts
ist normalverteilt mit $\mu = 250$ und $\sigma = 10$.

Bestimmen Sie die Wahrscheinlichkeit, dass der Inhalt der Tüte
\end{example}
\paragraph{a)} eine Masse von mindestens 230g hat
\begin{marginfigure}
\resizebox{\linewidth}{!}{
\begin{tikzpicture}
\begin{axis}[every axis plot post/.append style={
	mark=none,domain=20:30,samples=50,smooth}, % All plots: from -2:2, 50 samples, smooth, no marks
	enlarge y limits={value=0.2,upper},
	enlarge x limits=false,
	ticks=none,] % extend the axes a bit to the right and top
	\addplot+[blue,thick,name path=A] {gauss(25,1)};
	\addplot+[draw=none,name path=B] {0};     % “fictional” curve
	\addplot+[blue!20!white] fill between[of=A and B,soft clip={domain=23:30}]; % filling
	\addplot[blue,dashed,thick]
      coordinates {(23,0) (23,0.14)}
      node[above=0pt] {$a$};
\end{axis}
\end{tikzpicture}
}
\caption{Zu Teilaufgabe a)}
\end{marginfigure}
\addtolength{\jot}{1em}
\begin{align*}
&\mu = 250; \sigma = 10 &\parbox[t]{0.5\linewidth}{Erwartungswert; Standardabweichung} \\
&a = 230; b = \infty &\parbox[t]{0.5\linewidth}{Per \texttt{Normalcdf} ausrechnen;\\ \texttt{LOWERBnd = 230}; \texttt{UPPERBnd = 1E99}} \\
&P(X \geq a) = \int_a^b\varphi_{\mu; \sigma}(x)dx &\parbox[t]{0.5\linewidth}{Wahrscheinlichkeit berechnen} \\
&P(X \geq 230) = 0,977 &\parbox[t]{0.5\linewidth}{Wahrscheinlichkeit für mindestens 230g beträgt $0,997$}
\end{align*}
\paragraph{b)} eine Masse von weniger als 250g hat
\begin{marginfigure}
\resizebox{\linewidth}{!}{
\begin{tikzpicture}
\begin{axis}[every axis plot post/.append style={
	mark=none,domain=20:30,samples=50,smooth}, % All plots: from -2:2, 50 samples, smooth, no marks
	enlarge y limits={value=0.2,upper},
	enlarge x limits=false,
	ticks=none,] % extend the axes a bit to the right and top
	\addplot+[blue,thick,name path=A] {gauss(25,1)};
	\addplot+[draw=none,name path=B] {0};     % “fictional” curve
	\addplot+[blue!20!white] fill between[of=A and B,soft clip={domain=20:25}]; % filling
	\addplot[blue,dashed,thick]
      coordinates {(25,0) (25,0.45)}
      node[above=0pt] {$b$};
\end{axis}
\end{tikzpicture}
}
\caption{Zu Teilaufgabe b)}
\end{marginfigure}
\begin{align*}
&\mu = 250; \sigma = 10 &\parbox[t]{0.5\linewidth}{Erwartungswert; Standardabweichung} \\
&a = \infty; b = 250 &\parbox[t]{0.5\linewidth}{Per \texttt{Normalcdf} ausrechnen;\\ \texttt{LOWERBnd = -1E99}; \texttt{UPPERBnd = 250}} \\
&P(X < a) = \int_a^b\varphi_{\mu; \sigma}(x)dx &\parbox[t]{0.5\linewidth}{Wahrscheinlichkeit berechnen} \\
&P(X < 250) = 0,5 &\parbox[t]{0.5\linewidth}{Wahrscheinlichkeit für weniger als 250g beträgt $0,5$}
\end{align*}
\paragraph{c)} eine Masse zwischen 245g und 255g hat
\begin{marginfigure}
\resizebox{\linewidth}{!}{
\begin{tikzpicture}
\begin{axis}[every axis plot post/.append style={
	mark=none,domain=20:30,samples=50,smooth}, % All plots: from -2:2, 50 samples, smooth, no marks
	enlarge y limits={value=0.2,upper},
	enlarge x limits=false,
	ticks=none,] % extend the axes a bit to the right and top
	\addplot+[blue,thick,name path=A] {gauss(25,1)};
	\addplot+[draw=none,name path=B] {0};     % “fictional” curve
	\addplot+[blue!20!white] fill between[of=A and B,soft clip={domain=24.5:25.5}]; % filling
	\addplot[blue,dashed,thick]
      coordinates {(24.5,0) (24.5,0.45)}
      node[above=0pt] {$a$};
  \addplot[blue,dashed,thick]
      coordinates {(25.5,0) (25.5,0.45)}
      node[above=0pt] {$b$};
\end{axis}
\end{tikzpicture}
}
\caption{Zu Teilaufgabe c)}
\end{marginfigure}
\begin{align*}
&\mu = 250; \sigma = 10 &\parbox[t]{0.5\linewidth}{Erwartungswert; Standardabweichung} \\
&a = 245; b = 255 &\parbox[t]{0.5\linewidth}{Per \texttt{Normalcdf} ausrechnen;\\ \texttt{LOWERBnd = 245}; \texttt{UPPERBnd = 255}} \\
&P(a \leq X \leq b) = \int_a^b\varphi_{\mu; \sigma}(x)dx &\parbox[t]{0.5\linewidth}{Wahrscheinlichkeit berechnen} \\
&P(245 \leq X \leq 255) = 0,383 &\parbox[t]{0.5\linewidth}{Wahrscheinlichkeit für mindestens 245g und höchstens 255 beträgt $0,383$}
\end{align*}


\end{document}
